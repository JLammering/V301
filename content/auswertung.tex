\input{Praeambel.tex}
\begin{document}
  \section{Auswertung}
  \subsection{Messergebnisse}

  In den Tabellen \ref{tab:U1} und \ref{tab:U2} sind jeweils
  Klemmenspannung $U_k$ und Stromstärke $I$
  der ersten und zweiten Schaltung aufgetragen. Der Belastungswiderstand $R_A$
  lag im Bereich $ \SI{0}{\ohm} \leq R_A \leq \SI{50}{\ohm} $.
  Die Messfehler betrugen für beide Messungen $U_\text{k,m} = \SI{+-0.06}{\V}$
  und $I_m = \pm \SI{0.003}{\A}$.
  Es ist zu beachten, dass die ersten beiden Werte der Stromstärke bei
  der ersten Messung und die ersten drei Werte der Stromstärke bei der zweiten
  Messung einen 10-mal so großen Messfehler haben.

  \begin{table}[h]
    \begin{minipage}{0.45\textwidth}
    \centering
    \begin{tabular}{S S}
      \toprule
      $U_k/\si{\V}$ & $I/\si{\A}$ \\
      \midrule
      0.60 & 0.230 \\
      0.70 & 0.140 \\
      0.93 & 0.086 \\
      1.05 & 0.065 \\
      1.13 & 0.052 \\
      1.16 & 0.045 \\
      1.21 & 0.038 \\
      1.22 & 0.034 \\
      1.25 & 0.030 \\
      1.26 & 0.028 \\
      \bottomrule
    \end{tabular}
    \label{tab:U1}
    \caption{Messwerte von Schaltung 1}
    \end{minipage}\hfill
    \begin{minipage}{0.45\textwidth}
      \centering
      \begin{tabular}{S S}
        \toprule
        $U_k/\si{\V}$ & $I/\si{\A}$ \\
        \midrule
        3.25 & 0.330 \\
        2.49 & 0.185 \\
        2.17 & 0.122 \\
        1.99 & 0.096 \\
        1.88 & 0.075 \\
        1.80 & 0.062 \\
        1.76 & 0.054 \\
        1.73 & 0.048 \\
        1.69 & 0.042 \\
        1.67 & 0.038 \\
        \bottomrule
      \end{tabular}
      \label{tab:U2}
      \caption{Messwerte von Schaltung 2}
    \end{minipage}
  \end{table}

  Zur lineare Ausgleichsrechnung wurde die Funktion linregress aus scipy.stats
  benutzt. Es wurde jeweils Steigung, Y-Achsen-Abschnitt und die
  Messungenauigkeit zurückgegeben.

  \newpage

  \begin{figure}[h]
    \includegraphics[width = \textwidth]{plot1.pdf}
    \label{fig:U1}
    \caption{Messwerte und Fehler $U_\text{k,1}(I)$}
  \end{figure}

  In dem Graphen \ref{fig:U1} sind die Messwerte der Tabelle
  \ref{tab:U1} und ihre lineare Regression
  $U_\text{k,1}(I)$ aufgetragen.
  Das erste Wertepaar $\SI{0.60}{\V}/ \SI{0.230}{\A}$
  wurde allerdings nicht mit in den Plot einbezogen,
  da es sehr stark von den anderen Wertepaaren abweicht und
  den Messfehler der linearen Regression stark vergrößert.
  Folgende Werte wurden bei der Ausgleichsrechnung zurückgegeben:
  \begin{align}
    m_1 & = \num{-0.196 +- 0.005} & b_1 & = 0.273
  \end{align}
  Mit
  \begin{equation}
    U_\text{k,1}(I) = U_\text{0,1} - I \cdot R_\text{i,1}
  \end{equation}
  für die erste Schaltung ist
  \begin{equation}
    U_\text{0,1} = b_1 \si{\V} = \SI{0.273}{\V}
  \end{equation}
  und
  \begin{equation}
    R_\text{i,1} = -m_1 \si{\ohm} = \SI{0.196 +- 0.005}{\ohm}.
  \end{equation}

  \newpage

  \begin{figure}[h]
    \includegraphics[width = \textwidth]{plot2.pdf}
    \label{fig:U2}
    \caption{Messwerte und Fehler $U_\text{k,2}(I)$}
  \end{figure}

  In dem Graphen \ref{fig:U2} sind die Messwerte der Tabelle
  \ref{tab:U2} und ihre lineare Regression
  $U_\text{k,2}(I)$ aufgetragen.
  Bei der Ausgleichungsrechnung wuden folgene Werte zurückgegeben:
  \begin{align}
    m_2 & = \num{5.44 +- 0.06} & b_2 & = 1.47
  \end{align}
  Da in Schaltung 2 eine Gegenspannung, die in etwa $\SI{2}{\V}$ größer als
  $U_0$ ist, mit eingebaut wurde, fließt der Strom in umgekehrte Richtung.
  Die Klemmenspannung ist daher
  \begin{equation}
    U_\text{k,2}(I) = U_\text{0,2} + I \cdot R_\text{i,2}.
  \end{equation}
  Für Schaltung 2 ergeben sich dann Leerlaufspannung
  \begin{equation}
    U_\text{0,2} = b_2 \si{\V} = \SI{1.47}{\V}
  \end{equation}
  und Innenwiderstand
  \begin{equation}
    R_\text{i,2} = m_2 \si{\ohm} = \SI{5.44 +- 0.06}{\ohm}.
  \end{equation}

  \newpage

  In den Tabellen \ref{tab:U3a} und \ref{tab:U3b} sind jeweils
  Klemmenspannung $U_K$ und Stromstärke $I$
  der Schaltungen 3a und 3b aufgetragen. Der Belastungswiderstand $R_A$
  lag bei 3a im Bereich $ \SI{20}{\ohm} \leq R_A \leq \SI{250}{\ohm} $ und bei
  3b im Bereich $ \SI{0.1}{\kilo\ohm} \leq R_A \leq \SI{5}{\kilo\ohm} $ .
  Die Messfehler beim Rechteck-Ausgang 3a betrugen
  $U_\text{k,m} = \SI{+-0.03}{\V}$ und $I_m = \pm \SI{0.3}{\milli\A}$ und die
  beim Sinus-Ausgang 3b $U_\text{k,m} = \SI{+-0.09}{\V}$ und
  $I_m = \pm \SI{0.09}{\milli\A}$.

  \begin{table}[h]
    \begin{minipage}{0.45\textwidth}
    \centering
    \begin{tabular}{S S}
      \toprule
      $U_k/\si{\V}$ & $I/\si{\milli\A}$ \\
      \midrule
      0.165 & 6.4 \\
      0.215 & 5.5 \\
      0.275 & 4.3 \\
      0.318 & 3.5 \\
      0.341 & 3.0 \\
      0.361 & 2.6 \\
      0.378 & 2.3 \\
      0.393 & 2.1 \\
      0.403 & 1.9 \\
      0.412 & 1.7 \\
      \bottomrule
    \end{tabular}
    \label{tab:U3a}
    \caption{Messwerte von Schaltung 1}
    \end{minipage}\hfill
    \begin{minipage}{0.45\textwidth}
      \centering
      \begin{tabular}{S S}
        \toprule
        $U_k/\si{\V}$ & $I/\si{\milli\A}$ \\
        \midrule
        0.73 & 2.55 \\
        1.22 & 1.84 \\
        1.55 & 1.26 \\
        1.76 & 0.93 \\
        1.89 & 0.71 \\
        1.96 & 0.59 \\
        2.01 & 0.50 \\
        2.06 & 0.45 \\
        2.08 & 0.41 \\
        2.10 & 0.37 \\
        \bottomrule
      \end{tabular}
      \label{tab:U3a)}
      \caption{Messwerte von Schaltung 2}
    \end{minipage}
  \end{table}

  \newpage

  \begin{figure}[h]
    \includegraphics[width = \textwidth]{plot3a.pdf}
    \label{fig:U3a}
    \caption{Messwerte und Fehler $U_\text{k,3a}(I)$}
  \end{figure}

  In dem Graphen \ref{fig:U3a} sind die Messwerte der Tabelle
  \ref{tab:U3a} und ihre lineare Regression
  $U_\text{k,3a}(I)$ aufgetragen.
  Folgende Werte wurden bei der Ausgleichsrechnung ausgegeben:
  \begin{align}
    m_\text{3a} & = \num{-19.1 +- 0.2} & b_\text{3a} & = 9.567
  \end{align}
  Mit
  \begin{equation}
    U_\text{k,3a}(I) = U_\text{0,3a} - I \cdot R_\text{i,3a}
  \end{equation}
  für die erste Schaltung ist
  \begin{equation}
    U_\text{0,3a} = b_\text{3a} \si{\V} = \SI{9.567}{\V}
  \end{equation}
  und
  \begin{equation}
    R_\text{i,3a} = -m_\text{3a} \si{\ohm} = \SI{0.196 +- 0.005}{\ohm}.
  \end{equation}

  \newpage

  \begin{figure}[h]
    \includegraphics[width = \textwidth]{plot3b.pdf}
    \label{fig:U3b}
    \caption{Messwerte und Fehler $U_\text{k,3b}(I)$}
  \end{figure}

  In dem Graphen \ref{fig:U3b} sind die Messwerte der Tabelle
  \ref{tab:U3b} und ihre lineare Regression
  $U_\text{k,3b}(I)$ aufgetragen.
  Folgende Werte wurden bei der Ausgleichsrechnung ausgegeben (umgerechnet von
  \si{\milli\A} in \si{\A}):
  \begin{align}
    m_\text{3b} & = \num{-621(6)e2} & b_\text{3b} & = 2.333
  \end{align}
  Mit
  \begin{equation}
    U_\text{k,3b}(I) = U_\text{0,3b} - I \cdot R_\text{i,3b}
  \end{equation}
  für die erste Schaltung ist
  \begin{equation}
    U_\text{0,3b} = b_\text{3b} \si{\V} = \SI{2.333}{\V}
  \end{equation}
  und
  \begin{equation}
    R_\text{i,3b} = -m_\text{3b} \si{\ohm} = \SI{-621(6)e2}{\ohm}.
  \end{equation}

  \newpage

  Wenn man in der Schaltung 1 das Voltmeter an die Stelle H
  setzen würde, würden die Messwerte der Klemmenspannung $U_k$ und der
  Stromstärke $I$ verfälscht werden. Das Voltmeter hat zwar einen großen
  Widerstand $R_V = \SI{10}{\mega\ohm}$, aber dennoch fließt ein kleiner Strom
  über diesen Pfad. Das Amperemeter würde also die Summe aus dem
  Klemmen-Strom $I_k$ und dem Voltmeter-Strom $I_V$ messen und nicht einfach
  nur den gesuchten Klemmen-Strom $I_k$.
  Zudem misst das Voltmeter nur die Potentialdifferenz an dem Außenwiderstand
  $R_A$ und nicht zusätzlich die minimale Spannung am Amperemeter. Dadurch
  sind die Werte des Innenwiderstandes $R_I$ ebenfalls verfälscht.
\end{document}
