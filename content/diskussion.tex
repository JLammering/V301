\section{Diskussion}
\label{sec:Diskussion}

Wenn man in der Schaltung \ref{fig:Schaltung1} das Voltmeter an die Stelle H
setzen würde, würden die Messwerte der Klemmenspannung $U_k$ und der
Stromstärke $I$ verfälscht werden. Das Voltmeter hat zwar einen großen
Widerstand $R_V = \SI{10}{\mega\ohm}$, aber dennoch fließt ein kleiner Strom
über diesen Pfad. Das Amperemeter würde also die Summe aus dem
Klemmen-Strom $I_k$ und dem Voltmeter-Strom $I_V$ messen und nicht einfach
nur den gesuchten Klemmen-Strom $I_k$.
Zudem misst das Voltmeter nur die Potentialdifferenz an dem Außenwiderstand
$R_A$ und nicht zusätzlich die minimale Spannung am Amperemeter. Dadurch
sind die Werte des Innenwiderstandes $R_I$ ebenfalls verfälscht.

In Abbildung \ref{fig:U5} sind die Messwerte aus Tabelle \ref{tab:U1}
mit $U_k \cdot I$ gegen $\frac{U_k}{I}$ und der Graph der Funktion $N(R_A)$
aufgetragen. Die Funktionsgleichung von $N(R_A)$ wurde über die Formel
\begin{equation}
  I \cdot R_A = U_0 - I \cdot R_i
\end{equation}
bstimmt. Es wurde nach der Stromstärke $I$ umgestellt und in die Gleichung
\begin{equation}
  N(R_A) = I^2 \cdot R_A
\end{equation}
eingesetzt, so dass
\begin{equation}
  N(R_A) = \frac{U_0^2 \cdot R_A^2}{(R_i + R_A)^2}.
\end{equation}
Die Messwerte weichen stark von dem Plot ab.

\begin{figure}[h]
  \includegraphics[width = \textwidth]{build/plot5.pdf}
  \caption{Messwerte und Leistungskurve $N(R_A)$}
  \label{fig:U5}
\end{figure}
