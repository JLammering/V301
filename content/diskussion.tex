\section{Diskussion}
\label{sec:Diskussion}

Wenn man in der Schaltung \ref{fig:Schaltung1} das Voltmeter an die Stelle H
setzen würde, würden die Messwerte der Klemmenspannung $U_k$ und der
Stromstärke $I$ verfälscht werden. Das Voltmeter hat zwar einen großen
Widerstand $R_V = \SI{10}{\mega\ohm}$, aber dennoch fließt ein kleiner Strom
über diesen Pfad. Das Amperemeter würde also die Summe aus dem
Klemmen-Strom $I_k$ und dem Voltmeter-Strom $I_V$ messen und nicht einfach
nur den gesuchten Klemmen-Strom $I_k$.
Zudem misst das Voltmeter nur die Potentialdifferenz an dem Außenwiderstand
$R_A$ und nicht zusätzlich die minimale Spannung am Amperemeter. Dadurch
sind die Werte des Innenwiderstandes $R_I$ ebenfalls verfälscht.
