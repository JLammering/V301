\section{Theorie}
\label{sec:Theorie}

Eine Spannungsquelle ist ein elektrisches Bauteil, dass über einen Zeitraum, der endlich ist,
eine konstante Leistung liefert.

Die Spannung, die an den Ausgängen der Spannungsquelle abgegriffen wird, wird
mit Klemmenspannung $U_\text{K}$ bezeichnet. Wenn kein Strom entnommen wird, bezeichnet
man sie mit Leerlaufspannung $U_\text{0}$. Sobald nun aber ein Stromverbraucher
angeschlossen wird, sinkt $U_K$ auf einen Wert unterhalb $U_\text{0}$ nach der Gleichung
\eqref{eqn:Klemmenspannung}.

\begin{equation}
  U_\text{K}(I) = U_\text{0} - I \cdot R_\text{i}
  \label{eqn:Klemmenspannung}
\end{equation}

Jede Spannungsquelle besitzt einen Innenwiderstand $R_\text{i}$, der bewirkt, dass
ihr keine unendliche Leistung entnommen werden kann.

\begin{equation}
  N = I^2 \cdot R_\text{a}
  \label{eqn:Leistung}
\end{equation}

Mit Gleichung \eqref{eqn:Leistung} kann die Leistung $N$ in Abhängigkeit von der Stromstärke $I$
und dem äußeren Widerstand $R_\text{a}$ berechnet werden.

Wenn man nun aber Gleichung
\eqref{eqn:Klemmenspannung} nach $I$ umformt und in \eqref{eqn:Leistung} einsetzt
zeigt sich, dass die Leistung nur noch von $R_\text{a}$ abhängt nach Gleichung
\eqref{eqn:Leistungmitra}.

\begin{equation}
  N = \frac{(U_\text{0})^2}{(R_\text{i} + R_\text{a})^2} \cdot R_\text{a}
  \label{eqn:Leistungmitra}
\end{equation}

Diese Funktion $N(R_\text{a})$ duchläuft ein Maximum, wie sich in Abbildung
\ref{fig:U5} zeigt. An dieser Stelle spricht man von Leistungsanpassung.

Außerdem muss noch erwähnt werden, dass der Innenwiderstand bei elektrischen
Generatoren durch Gleichung \eqref{eqn:Innenwiderstand} ausgedrückt wird.

\begin{equation}
  R_\text{i} = \frac{\text{d}U_\text{K}}{\text{d}I}
  \label{eqn:Innenwiderstand}
\end{equation}

\newpage
