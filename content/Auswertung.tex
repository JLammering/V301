  \section{Auswertung}
  \subsection{Messergebnisse}

  In den Tabellen \ref{tab:Uteins} und \ref{tab:Utzwei} sind jeweils
  Klemmenspannung $U_k$ und Stromstärke $I$
  der ersten und zweiten Schaltung aufgetragen.
  Die Messfehler betragen für beide Messungen $U_\text{k,m} = \SI{+-0.06}{\V}$
  und $I_m = \pm \SI{0.003}{\A}$.
  Es ist zu beachten, dass die ersten beiden Werte der Stromstärke bei
  der ersten Messung und die ersten drei Werte der Stromstärke bei der zweiten
  Messung einen 10-mal so großen Messfehler haben.

  \begin{table}[h]
    \begin{minipage}{0.45\textwidth}
    \centering
    \caption{Messwerte von Schaltung 1}
    \label{tab:Uteins}
    \begin{tabular}{S S}
      \toprule
      $U_k/\si{\V}$ & $I/\si{\A}$ \\
      \midrule
      0.60 & 0.230 \\
      0.70 & 0.140 \\
      0.93 & 0.086 \\
      1.05 & 0.065 \\
      1.13 & 0.052 \\
      1.16 & 0.045 \\
      1.21 & 0.038 \\
      1.22 & 0.034 \\
      1.25 & 0.030 \\
      1.26 & 0.028 \\
      \bottomrule
    \end{tabular}
    \end{minipage}\hfill
    \begin{minipage}{0.45\textwidth}
      \centering
      \caption{Messwerte von Schaltung 2}
      \label{tab:Utzwei}
      \begin{tabular}{S S}
        \toprule
        $U_k/\si{\V}$ & $I/\si{\A}$ \\
        \midrule
        3.25 & 0.330 \\
        2.49 & 0.185 \\
        2.17 & 0.122 \\
        1.99 & 0.096 \\
        1.88 & 0.075 \\
        1.80 & 0.062 \\
        1.76 & 0.054 \\
        1.73 & 0.048 \\
        1.69 & 0.042 \\
        1.67 & 0.038 \\
        \bottomrule
      \end{tabular}
      \end{minipage}
  \end{table}

  Bei der ersten Messung der Leerlaufspannung der Monozelle beträgt die
  gemessene Klemmenspannung

  \begin{equation}
    U_\text{k,0} = \SI{1.40(6)}{\V}.
  \end{equation}

  Der Eingangswiderstand $R_V$ des Voltmeters ist

  \begin{equation}
    R_V = \SI{10}{\mega\ohm}
  \end{equation}

  Zur lineare Ausgleichsrechnung wird die Funktion $linregress$ aus $scipy.stats$
  benutzt. Dabei wird jeweils Steigung, Y-Achsen-Abschnitt und die
  Messungenauigkeit zurückgegeben.

  \newpage

  \begin{figure}[h]
    \centering
    \includegraphics[width = \textwidth]{plot1.pdf}
    \caption{Messwerte und Fehler, Monozelle,  $U_\text{k,1}(I)$}
    \label{fig:Ueins}
  \end{figure}

  In dem Graphen von Abbildung \ref{fig:Ueins} sind die Messwerte der Tabelle
  \ref{tab:Uteins} und ihre lineare Regression
  $U_\text{k,1}(I)$ aufgetragen.
  Das erste Wertepaar $\SI{0.60}{\V}/ \SI{0.230}{\A}$
  wird allerdings nicht mit in die lineare Regression einbezogen,
  da es sehr stark von den anderen Wertepaaren abweicht und
  den Messfehler der Ausgleichsgeraden stark vergrößert.
  Folgende Werte werden bei der Ausgleichsrechnung zurückgegeben:

  \begin{align}
    m_1 & = \SI{-5.1(1)}{\V\per\A} & b_1 & = \SI{1.4}{\V}
  \end{align}

  Mit

  \begin{equation}
    U_\text{k,1}(I) = U_\text{0,1} - I \cdot R_\text{i,1}
  \end{equation}

  für die erste Schaltung ist

  \begin{equation}
    U_\text{0,1} = b_1 = \SI{1.4}{\V}
  \end{equation}

  und

  \begin{equation}
    R_\text{i,1} = -m_1 = \SI{5.1(1)}{\ohm}.
  \end{equation}

  \newpage

  \begin{figure}[h]
    \centering
    \includegraphics[width = \textwidth]{plot2.pdf}
    \caption{Messwerte und Fehler, Monozelle mit Gegenspannung, $U_\text{k,2}(I)$}
    \label{fig:Uzwei}
  \end{figure}

  In dem Graphen von Abbildung \ref{fig:Uzwei} sind die Messwerte der Tabelle
  \ref{tab:Utzwei} und ihre lineare Regression
  $U_\text{k,2}(I)$ aufgetragen.
  Bei der Ausgleichungsrechnung werden folgene Werte zurückgegeben:

  \begin{align}
    m_2 & = \num{5.44 +- 0.06} & b_2 & = 1.47
  \end{align}

  Da in Schaltung 2 eine Gegenspannung, die in etwa $\SI{2}{\V}$ größer als
  $U_0$ ist, mit eingebaut wurde, fließt der Strom in umgekehrte Richtung.
  Die Klemmenspannung ist daher

  \begin{equation}
    U_\text{k,2}(I) = U_\text{0,2} + I \cdot R_\text{i,2}.
  \end{equation}

  Für Schaltung 2 ergeben sich dann Leerlaufspannung

  \begin{equation}
    U_\text{0,2} = b_2 \si{\V} = \SI{1.47}{\V}
  \end{equation}

  und Innenwiderstand

  \begin{equation}
    R_\text{i,2} = m_2 \si{\ohm} = \SI{5.44 +- 0.06}{\ohm}.
  \end{equation}

  \newpage

  In den Tabellen \ref{tab:Utdreia} und \ref{tab:Utdreib} sind jeweils
  Klemmenspannung $U_K$ und Stromstärke $I$
  der Schaltungen 3a und 3b aufgetragen.
  Die Messfehler beim Rechteck-Ausgang 3a betragen
  $U_\text{k,m} = \SI{+-0.03}{\V}$ und $I_m = \pm \SI{0.3}{\milli\A}$ und die
  beim Sinus-Ausgang 3b $U_\text{k,m} = \SI{+-0.09}{\V}$ und
  $I_m = \pm \SI{0.09}{\milli\A}$.

  \begin{table}[h]
    \begin{minipage}{0.45\textwidth}
    \centering
    \caption{Messwerte von Schaltung 1}
    \label{tab:Utdreia}
    \begin{tabular}{S S}
      \toprule
      $U_k/\si{\V}$ & $I/\si{\milli\A}$ \\
      \midrule
      0.165 & 6.4 \\
      0.215 & 5.5 \\
      0.275 & 4.3 \\
      0.318 & 3.5 \\
      0.341 & 3.0 \\
      0.361 & 2.6 \\
      0.378 & 2.3 \\
      0.393 & 2.1 \\
      0.403 & 1.9 \\
      0.412 & 1.7 \\
      \bottomrule
    \end{tabular}
    \end{minipage}\hfill
    \begin{minipage}{0.45\textwidth}
      \centering
      \caption{Messwerte von Schaltung 2}
      \label{tab:Utdreib}
      \begin{tabular}{S S}
        \toprule
        $U_k/\si{\V}$ & $I/\si{\milli\A}$ \\
        \midrule
        0.73 & 2.55 \\
        1.22 & 1.84 \\
        1.55 & 1.26 \\
        1.76 & 0.93 \\
        1.89 & 0.71 \\
        1.96 & 0.59 \\
        2.01 & 0.50 \\
        2.06 & 0.45 \\
        2.08 & 0.41 \\
        2.10 & 0.37 \\
        \bottomrule
      \end{tabular}
    \end{minipage}
  \end{table}

  \newpage

  \begin{figure}[h]
    \centering
    \includegraphics[width = \textwidth]{plot3a.pdf}
    \caption{Messwerte und Fehler, Rechtecks-Ausgang, $U_\text{k,3a}(I)$}
    \label{fig:Udreia}
  \end{figure}

  In dem Graphen von Abbildung \ref{fig:Udreia} sind die Messwerte der Tabelle
  \ref{tab:Utdreia} und ihre lineare Regression
  $U_\text{k,3a}(I)$ aufgetragen.
  Folgende Werte werden bei der Ausgleichsrechnung ausgegeben:

  \begin{align}
    m_\text{3a} & = \SI{-0.0522(5)}{\V\per\A} & b_\text{3a} & = \SI{0.5001}{\V}
  \end{align}

  Mit

  \begin{equation}
    U_\text{k,3a}(I) = U_\text{0,3a} - I \cdot R_\text{i,3a}
  \end{equation}

  für die erste Schaltung ist

  \begin{equation}
    U_\text{0,3a} = b_\text{3a} = \SI{0.5001}{\V}
  \end{equation}

  und

  \begin{equation}
    R_\text{i,3a} = -m_\text{3a} = \SI{0.0522(5)}{\ohm}.
  \end{equation}

  \newpage

  \begin{figure}[h]
    \centering
    \includegraphics[width = \textwidth]{plot3b.pdf}
    \caption{Messwerte und Fehler, Sinus-Ausgang $U_\text{k,3b}(I)$}
    \label{fig:Udreib}
  \end{figure}

  In dem Graphen von Abbildung \ref{fig:Udreib} sind die Messwerte der Tabelle
  \ref{tab:Utdreib} und ihre lineare Regression
  $U_\text{k,3b}(I)$ aufgetragen.
  Folgende Werte werden bei der Ausgleichsrechnung ausgegeben (umgerechnet von
  \si{\milli\A} in \si{\A}):

  \begin{align}
    m_\text{3b} & = \num{-0.621(6)} & b_\text{3b} & = 2.333
  \end{align}

  Mit

  \begin{equation}
    U_\text{k,3b}(I) = U_\text{0,3b} - I \cdot R_\text{i,3b}
  \end{equation}

  für die erste Schaltung ist

  \begin{equation}
    U_\text{0,3b} = b_\text{3b} \si{\V} = \SI{2.333}{\V}
  \end{equation}

  und

  \begin{equation}
    R_\text{i,3b} = -m_\text{3b} \si{\ohm} = \SI{-0.621(6)}{\ohm}.
  \end{equation}

  \newpage

  Bei der Messung der Leerlaufspannung $U_0$ mit einem hochohmigen Voltmeter
  wurde der Vorteil ausgenutzt, dass bei einer niedrige Stromstärke die
  gemessene Klemmenspanung $U_k$ etwa der Leerlaufspannung entspricht, da
  der Anteil des Innenwiderstandes nahe Null ist.
  Es existiert dennoch ein geringer Fehler, der im Folgenden bestimmt wird.
  Mit

  \begin{equation}
    I = \frac{U_\text{k,0}}{R_V} = \SI{1.40(6)e-7}{\A}
  \end{equation}

  und dem in \ref{fig:Ueins} bestimmten Innenwiderstand

  \begin{equation}
    R_\text{i,1} = \SI{5.1(1)}{\ohm}
  \end{equation}

  ergibt sich der systematische Fehler ohne Abweichung mit der Formel

  \begin{equation}
    U_\text{0,0} - U_\text{k,0} = I \cdot R_i = \SI{7.14e-7}{\V}
  \end{equation}

  Für die Bestimmung der Abweichung muss die Messfehlerfortpflanzung
  beachtet werden. Über die Formel zur Gaußschen Fehlerfortpflanzung

  \begin{equation}
    \phi = \sqrt{\sum_{i=1}^{N} \biggl(\frac{\partial f}{\partial x_i}\biggr)^2
    \cdot \sigma_i^2}
  \end{equation}

  ergibt sich die neue Abweichung

  \begin{equation}
    \phi = \sqrt{\biggl(\frac{\partial(I \cdot R_i)}{\partial I}\biggr)^2
    \cdot \sigma_I^2
    + \biggl(\frac{\partial(I \cdot R_i)}{\partial R_i}\biggr)^2 \cdot
    \sigma_\text{R}^2} \si{\V}
    = \SI{0.3365e-7}{\V}
  \end{equation}

  Die Leerlaufspannung $U_\text{0,0}$ mit systematischem Fehler und der
  resultierenden Abweichung beträgt dann

  \begin{equation}
    U_\text{0,0} = \SI{7.1(3)e-7}{\V}.
  \end{equation}

  \newpage

  \begin{figure}[h]
    \includegraphics[width = \textwidth]{build/plot5.pdf}
    \caption{Messwerte und Leistungskurve $N(R_A)$}
    \label{fig:U5}
  \end{figure}

  In Abbildung \ref{fig:U5} sind die Messwerte aus Tabelle \ref{tab:Uteins}
  mit $U_k \cdot I$ gegen $\frac{U_k}{I}$ und der Graph der Funktion $N(R_A)$
  aufgetragen. Die Funktionsgleichung von $N(R_A)$ wird über die Formel

  \begin{equation}
    I \cdot R_A = U_0 - I \cdot R_i
  \end{equation}

  bstimmt. Es wird nach der Stromstärke $I$ umgestellt und in die Gleichung

  \begin{equation}
    N(R_A) = I^2 \cdot R_A
  \end{equation}

  eingesetzt, so dass

  \begin{equation}
    N(R_A) = \frac{U_0^2 \cdot R_A}{(R_i + R_A)^2}.
  \end{equation}
